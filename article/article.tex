\documentclass{article}
\usepackage[french]{babel}
\usepackage[T1]{fontenc}
\usepackage{amsmath}
\usepackage{amsthm}
\usepackage{stmaryrd}
\usepackage{amsfonts}
\usepackage{amssymb}
\usepackage{graphicx}
\usepackage{subcaption}
\usepackage[a4paper, total={\textwidth, 23cm}]{geometry}
\usepackage{setspace}
\usepackage{bbold}
\usepackage{chngcntr}
\usepackage{xcolor}
\usepackage{hyperref}
\hypersetup{
	colorlinks=true,
	linkcolor=blue,
	citecolor=red,
	pdftitle={Rapport de Stage},
	pdfauthor={Etienne MARION},
	pdfpagemode=FullScreen,
}

\newcommand{\lean}[1]{\lstinline[language=lean]{#1}}

\definecolor{keywordcolor}{rgb}{0.7, 0.1, 0.1}   % red
\definecolor{tacticcolor}{rgb}{0.0, 0.1, 0.3}    % dark blue
\definecolor{commentcolor}{rgb}{0.4, 0.4, 0.4}   % grey
\definecolor{stringcolor}{rgb}{0.5, 0.3, 0.2}    % brown
\definecolor{symbolcolor}{rgb}{0.1, 0.2, 0.7}    % blue
\definecolor{sortcolor}{rgb}{0.1, 0.5, 0.1}      % green
\definecolor{attributecolor}{rgb}{0.7, 0.1, 0.1} % red
\definecolor{errorcolor}{rgb}{1, 0, 0}           % bright red

\usepackage{listings}
\def\lstlanguagefiles{lstlean.tex}
\lstloadlanguages{lean}
\lstset{language=lean}
\usepackage{scalefnt}

\newcommand{\sorry}[0]{\lean{sorry}}

\newcommand{\A}{\mathcal{A}}
\newcommand{\B}{\mathcal{B}}
\newcommand{\C}{\mathcal{C}}
\newcommand{\E}{\mathbb{E}}
\newcommand{\F}{\mathcal{F}}
\newcommand{\N}{\mathbb{N}}
\newcommand{\R}{\mathbb{R}}
\newcommand{\imp}{\Rightarrow}
\newcommand{\set}[1]{\left\{#1\right\}}
\newcommand{\inv}{^{-1}}
\newcommand{\eps}{\epsilon}
\newcommand{\type}{\texttt{Type}}
\newcommand{\prop}{\texttt{Prop}}
\newcommand{\prun}{\texttt{pr}_1}
\newcommand{\prde}{\texttt{pr}_2}
\newcommand{\inl}{\texttt{inl}}
\newcommand{\inr}{\texttt{inr}}
\newcommand{\skl}{\vspace{\baselineskip}}
\newcommand{\gui}[1]{\og{}#1\fg{}}
\newcommand{\Xge}[1]{X^{\ge#1}}
\newcommand{\Xle}[1]{X^{\le#1}}
\newcommand{\Xgt}[1]{X^{>#1}}
\newcommand{\Age}[1]{\mathcal{A}^{\ge#1}}
\newcommand{\Agt}[1]{\mathcal{A}^{>#1}}
\newcommand{\Ale}[1]{\mathcal{A}^{\le#1}}
\newcommand{\Xgtn}{X^{>n}}
\newcommand{\Xgtm}{X^{>m}}
\newcommand{\Xgtk}{X^{>k}}
\newcommand{\Xlen}{X^{\le n}}
\newcommand{\Alen}{\mathcal{A}^{\le n}}
\newcommand{\dx}{\mathrm{d}x}
\newcommand{\dy}{\mathrm{d}y}
\newcommand{\prth}[1]{\left(#1\right)}
\renewcommand{\empty}{\varnothing}
\newcommand{\tendl}[1]{\longrightarrow_{#1}}
\newcommand{\dbrack}[1]{\llbracket #1 \rrbracket}
\newcommand{\priv}{\,\backslash\,}
\newcommand{\restr}[1]{\vert_{#1}}
\newcommand{\mub}{\overline{\mu}}
\newcommand{\ox}{\otimes}
\renewcommand{\phi}{\varphi}
\newcommand{\piun}{{\pi_{\dbrack{1,n}}}}
\newcommand{\pizn}{{\pi_{\dbrack{0,n}}}}
\newcommand{\Xint}[1]{X^{\dbrack{#1}}}
\newcommand{\tost}{\to_*}
\newcommand{\ind}[1]{\mathbb{1}_{#1}}
\newcommand{\dmu}{\mathrm{d}\mu}

\newtheorem{thm}{Theorem}[section]
\newtheorem{lem}[thm]{Lemma}
\theoremstyle{definition}
\newtheorem{defi}[thm]{Definition}
\newtheorem{eg}[thm]{Example}
\theoremstyle{remark}
\newtheorem*{nota}{Notation}

\begin{document}

	\begin{center}
		{\Large\bf A Formalization of the Ionescu-Tulcea Theorem in mathlib} \\
		\vspace{1cm}
		Auteur : Etienne Marion \\
		\vspace{1cm}
		ENS de Lyon, 46, Allée d’Italie, 69007 Lyon, France \\
		\texttt{etienne.marion@ens-lyon.fr}
	\end{center}

	\vspace{0.5cm}

	\section*{Abstract}

	\section{Introduction}
	Being able to talk about the joint distribution of an infinite family of random variables is crucial in probability theory. For example, one often requires a family of independent random variables. The existence of such a family relies on the existence of an infinite product measure. Indeed, given $(\Omega_i, \F_i, \mu_i)_{i\in\iota}$ a family of probability spaces, the existence of the product measure $\bigotimes_{i\in\iota}\mu_i$ yields a new probability space $(\prod_{i\in\iota}\Omega_i, \bigotimes_{i\in\iota}\F_i, \bigotimes_{i\in\iota}\mu_i)$, and the projections $X_i : \prod_{j\in\iota}\Omega_j \to \Omega_i$ give the desired family. For another example, consider discrete-time Markov chains. Given a measurable space $(E, \A)$ and a Markov kernel $\kappa : E \to E$, one might want to build a sequence $(X_n)_{n\in\N}$ of random variables with values in $E$ such that the conditional distribution of $X_{n+1}$ given $X_0, ..., X_n$ is $\kappa(X_n,\cdot)$. Such objects are fundamental in probability theory: families of independent variables allow to build more complicated objects, such as Brownian motion, while discrete-time Markov chains form a huge class of stochastic processes which contains random walks for instance. It so happens that those objects always exist without any restrictions on the spaces we consider. This is a direct consequence of the Ionescu-Tulcea theorem.

\end{document}